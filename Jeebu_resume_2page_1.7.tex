%%%%%%%%%%%%%%%%%%%%%%%%%%%%%%Resume Latex Template Document%%%%%%%%%%%%
%%Author	: Jeebu Jacob Thomas
%%Institution 	: IIT Bombay
%%Date		: 21-08-2013

\documentclass[a4paper,10pt]{article}
\usepackage{anysize}
\usepackage{amsmath}
\usepackage{amssymb}
\usepackage{graphicx}
\usepackage[left=0.75in, right=0.75in, top=0.75in, bottom=0.75in, includefoot, headheight=13.6pt]{geometry}
\usepackage{color,graphicx}
\usepackage{verbatim}
\usepackage{hyperref}
\usepackage{stmaryrd}
\usepackage{multirow}
\usepackage{textcomp}
\usepackage[usenames,dvipsnames]{xcolor}


\usepackage{verbatim} 


\hypersetup{
    bookmarks=true,         % show bookmarks bar?
    unicode=false,          % non-Latin characters in Acrobat's bookmarks
    pdftoolbar=true,        % show Acrobat's toolbar?
    pdfmenubar=true,        % show Acrobat's menu?
    pdffitwindow=true,      % page fit to window when opened
    pdftitle={CV - Jeebu  Jacob Thomas (Digital)},    % title
    pdfauthor={Jeebu Jacob Thomas},     % author
    pdfsubject={Resume},   % subject of the document
    colorlinks=true,       % false: boxed links; true: colored links
    linkcolor=magenta,        % color of internal links
    citecolor=blue,        % color of links to bibliography
    filecolor=magenta,      % color of file links
    urlcolor=RoyalPurple           % color of external links
}

\definecolor{titleColor}{rgb}{0.85, 0.85, 0.85}

\begin{document}
%%%%%%%%% virtical space %%%%%%%%%%%%%%%%%
\paragraph{}
\textbf{ }
\vspace{1.7in}
%%%%%%%%%%AREAS OF INTEREST%%%%%%%%%%%%%%%%
%%%%%%%%%%%%%%%%%%%%%%%%%%%%%%%%%%%%%%%%%%%


\colorbox{titleColor}{\parbox{6.5in}{\textbf{AREAS OF INTEREST}}}
 \begin{itemize}
  \setlength{\itemsep}{1pt}
  \item {{  Digital VLSI Design, Processor Design, Hardware design using HDL}}
 \end{itemize}

%%%%%%%%%%___TECHNICAL SKILLS____%%%%%%%%
 \colorbox{titleColor}{\parbox{6.5in}{\textbf{TECHNICAL SKILLS}}}

 \begin{itemize}
  \setlength{\itemsep}{1pt}
  \item \textbf{{Tools / Softwares:}} \textit{Cadence-Virtuoso, Xilinx-ISE, ModelSim, NGSPICE, Magic, MATLAB, Keil, Altera Quartus, Labview, Xilinx Chipscope, Xilinx EDK, Microsoft Visual Studio, Eagle, Git}
  \item \textbf{{Programming Languages:}} \textit{VHDL, Verilog, Assembly Language (8085/8051), C/C++, Python, Bluespec System Verilog(BSV)}
  \item \textbf{{Github Profile:}} \url{https://github.com/jeebujacob}
 \end{itemize}
 
 %%%%%%%%WORK EXPERIENCE%%%%%%%%%%%%%%%%%%%%%%%%%%
 
 
  \colorbox{titleColor}{\parbox{6.5in}{\textbf{PROFESSIONAL EXPERIENCE}}}
   \begin{itemize}
    \item \textbf{Project Engineer, CDAC Thiruvananthapuram \qquad\qquad\qquad\quad\qquad\qquad Jul'2010-Jul'2011\\}
    CDAC is a premier R\&D Organization under Ministry of Communication and Information Technology, Govt of India.

    $\circ$  Worked on  Design of \textit{Multi-Processor System on Chip(MPSoC)} using VHDL \\
    $\circ$  Debugged and Implemented AHB bus architecture using VHDL on the the \textit{MPSoC}\\
    $\circ$  Designed and implemented Timer/Counter for the \textit{MPSoC}
    \item \textbf{Project Research Assistant, IIT Bombay 	\qquad\qquad\qquad\qquad\quad\qquad\qquad\qquad Jul'2011-Present}
     $\circ$ Worked on Remote-Triggered Laboratory on Modern Digital Design, a project under MHRD, Govt of India\\
    $\circ$ Designed, and materialized a working FPGA prototype board \textit{Xenon}, using Altera Cyclone IV FPGA  
    
 \end{itemize}
 


%%%%%%%%%%% ____PROJECTS___ %%%%%%%%%%%
 \colorbox{titleColor}{\parbox{6.5in}{\textbf{PROJECTS / SEMINAR}}}

 \begin{itemize}

\setlength{\itemsep}{1pt}
  \item \textbf{{M.Tech. Project}}\\
	\textbf{Hardware Accelerator for GF2 Matrix Multiplication }\qquad\qquad\qquad\ \textbf{Software Used}:Xilinx ISE\\
	{\textbf{Guide} - Prof. Sachin. B. Patkar\\}
	The objective of this Project is to study different FPGA based cosystems, and implement an efficient Matrix Multiplier for GF(2).
	Matrix Multiplication being a primitive in applications like cryptography is of prime importance in High Performance Computing. 
	Hence an FPGA based hardware accelerator becomes an important choice for the cause. 
	
 \textbf{\textit{Completed Work:}} 
   Implemented 32 bit and 64 bit GF2 Matrix Multiplier using \textit{SIRC(Software Interface for Reconfigurable Computing)}, a Microsoft based Interface framework
   for hardware-software cosystem. The results were compared against the existing software library \textit{M4RI}. 

%%%%%%%%%%%%%%%%%%%%%%%%%%%%%%%%%%


  \item \textbf{{B.Tech. Project: Infrared Tracking System using 8051}}  \qquad\qquad\qquad\qquad \textbf{Software Used}: Keil\\
        {\textbf{Guide} - Asst Prof. Jyothish Chandran\\}                                                   
	Implemented an Infrared tracking system for slow moving objects. The design includes an IR transmitter mounted on top of the 
	object to be tracked would transmit the signals in all possible directions. Two receivers, each mounted on top of stepper motors
	would rotate in opposite sense. Once the object is tracked, the angle swept by the receivers would help trilaterate the exact location of the object.
	In addition, an LED matrix designed would display an approximate location of the object.

  \item \textbf{{R\&D Project: Hardware-Software Co-system in FPGA}}\\
	{\textbf{Guide} - Prof. Sachin. B. Patkar \qquad\qquad\qquad\quad\qquad\qquad\qquad\quad\qquad\qquad\qquad\qquad\qquad\qquad\quad \\}
	Designed Partial Basis Builder, Transpose block and Outer Accumulator in \textit{BSV} for Solutions to System of linear Equations, and
	implemented a 32*32 bit matrix multiplier in EDK.
	
  \item \textbf{{M.Tech seminar: Study of Xilinx Embedded Development Kit}}\\
	{\textbf{Guide} - Prof. Sachin. B. Patkar \qquad\qquad\qquad\quad\qquad\qquad\qquad\quad\qquad\qquad\qquad\qquad\qquad\qquad\quad\\}
	Re-created and implemented, a Sparse Matrix-Vector multiplier using EDK

 \end{itemize}

%%%%%%%%%%___RELEVANT COURSES___%%%%%%%%%%%
 \colorbox{titleColor}{\parbox{6.5in}{\textbf{RELEVANT COURSES}}}\\
 
 \begin{tabular}{p{2in}p{2in}p{2.5in}}
%$\circ$ Digital VLSI Design &$\circ$ CMOS Analog VLSI Design &$\circ$ Mixed Signal VLSI Design\\

$\circ$ VLSI Design LAB &$\circ$ System Design &$\circ$ Foundations of VLSI CAD    \\
$\circ$ Digital VLSI Design &$\circ$ Mixed Signal VLSI Design &$\circ$ CMOS Analog VLSI Design\\
$\circ$VLSI Technology &$\circ$ Solid State Devices &$\circ$ Microelectronics Simulation Lab 
\end{tabular}\\


 %%%%%%%%%%___COURSE PROJ / SEMINAR___%%%%%%%%%%%
 \colorbox{titleColor}{\parbox{6.5in}{\textbf{RELEVANT COURSE PROJECTS}}}

 \begin{itemize}
  \setlength{\itemsep}{1pt}

  \item \textbf{{Implementation of \textit{MIPS} Architecture}} \textit{(Course: System Design)}\\
	Designed and Implemented single cycle(32 bit) and multi-cycle(8 bit)  MIPS architecture.
	Seventeen 32 bit wide instructions implemented includes R-type,I-type and J-type instructions.
   

  \item \textbf{{Design and write synthesizable description of Data processor for 8085 in VHDL}}\\ \textit{(Course: VLSI Design Lab)}\\
	Synthesizable descriptions of data processor for 8085 were designed and written in VHDL.  Data processor implements ALU and performs  all data movement, stack and ALU operations.
% 	This includes PUSH and POP operations involving the stack processor.
	
  \item \textbf{{Design and write synthesizable descriptions of Instruction processor for 8085 in Verilog}}\\ \textit{(Course: VLSI Design Lab)}\\
	Synthesizable descriptions of instruction processor for 8085 were designed and written in verilog.Instruction Processor is an FSM which
	fetches the next instruction from an address given by the contents of the 16 bit program counter(PC), and stores it in the operation instruction register. This circuit will also execute all instructions which update PC.

  \item \textbf{{Design \& Layout of a 3-8 decoder}}   \textit{(Course: Digital VLSI Design)}\\
	Designed a 3-8 decoder in \textit{NGSPICE} using CMOS logic, with defined total electrical effort. Design included forks with
	best possible combinations of inverters to achieve equal delay. Decoder was laid out in \textit{Magic} and post layout
	simulation results were compared.

  \item \textbf{{Graph Bipartitioning}} \textit{(Course: Foundations of VLSI CAD)}\\
	Implemented three algorithms for Graph Bipartitioning namely Fiduccia-Mattheyses, Kernighan-Lin Algorithm, and Spectral bisection
	in python.
	
%   \item \textbf{{8-point FFT Implementation in Bluespec}} \textit{(Course: Foundations of VLSI CAD)}\\
% 	Implemented multicycle version of 3 stage pipelined 8-point FFT module in Bluespec System Verilog(BSV). The results
% 	were verified with a test function.
% 	
   \item \textbf{{Design and simulation of switched capacitor filter}} \textit{(Course: Mixed Signal VLSI)}\\
 	Designed a switched capacitor filter in UMC 180nm technology. Achieved maximum passband attenuation of 1dB at passband corner frequency (20 Khz)
 	and minimum attenuation of 20dB at stopband corner frequency (40KHz) at a sampling frequency of 220 Khz.  
 \end{itemize}
%%%%%%%%%%___POSITIONS OF RESPONSIBILITIES____%%%%%%%%
 \hspace{0.2in}\colorbox{titleColor}{\parbox{6.5in}{\textbf{POSITIONS OF RESPONSIBILITIES}}}
 \begin{itemize}
   \item Cultural Secretary, Hostel 1, IIT Bombay(2012-13). Co-ordinated various events and festivals.
   \item Organized Photography Workshop, Hostel 1, IIT Bombay(2012). 
   \item Assistant Co-ordinator, \textit{Horizon}, Saintgits College of Engineering(2008).
 \end{itemize}
 %%%%%%%%%%AWARDS AND ACHIEVEMENTS%%%%%%%
 \hspace{0.2in}\colorbox{titleColor}{\parbox{6.5in}{\textbf{AWARDS AND ACHIEVEMENTS}}}
 \begin{itemize}
  \item Winner, \textit{Step'n'Synchro}, Dance Competition, Marthoma Residential School(2004)
  \item Winner, Review Speech , Cultural Competition Marthoma Residential School(2002)
  \item Second Runner up, High Jump, Annual Sports,  Marthoma Residential School(2004)
  \item Participated in Movie Making and Photography GC, IIT Bombay(2012) 
 \end{itemize}
 %%%%%%%%%%___HOBBY___%%%%%%%%%%%
 \hspace{0.2in}\colorbox{titleColor}{\parbox{6.5in}{\textbf{HOBBIES AND INTERESTS}}}
 \begin{itemize}
  \item Driving, Tennis, Ping Pong, Badminton, Movies, Photography
 \end{itemize}
\end{document}
