\documentclass[a4paper,10pt]{article}
\usepackage[utf8]{inputenc}

%opening
\title{Placement Preparation Guidelines for MicroElectronics Batch IIT Bombay}
\author{Jeebu Jacob Thomas(Mtech 2011-2014,IIT Bombay)}

\begin{document}

\maketitle


\section*{Topics to Study} 

\subsection*{Topic 1}
Start with the SetUp and Hold Time concepts. If you are ignorant about these topics, you are dead as far as the placements are concerned. 
VLSI companies play with these concepts at every point of their design, and hence this is the biggest expectation they have from you 

\subsection*{Topic 2}
Inverter Characteristics: This is very important.Make questions self and analyze the  
different ways questions can be asked to you. Very unexpected questions can be
expected from here. Also study the power-dissipation, and power-delay product concepts. 

\subsection*{Topic 3}
Finite State Machine ideas must be very clear. Questions from designing pattern detection to divider logics can be asked. Hence thorough understanding of FSM is a must

\subsection*{Topic 4}
Aptitude Preparation should never be taken lightly, and hence solve it regularly. It might be an edge to certain companies exams, where the technical problems are difficult to solve,
or you screwed up in those areas.

\subsection*{Topic 5}
Verilog Study. Syntax or the semantics are not important, rather the logic of the hardware you built, datatypes are more important and have a good understanding of them all

\subsection*{Topic 6}
Computer Architecture basics. Cache memory, cache hit cases and related topics

\subsection*{Topic 7} Basic C Skills, pointers etc

\section*{Things to be very careful in the Selection Procedure, based on the experiences of Seniors}

\subsection*{Point1}
Always try to be there in the selection procedure on time, or sometime before that. Attending the selection procedure, especially the interview at the earliest may have its own  advantages.
The perspective of the interviewers is to get the required number of eligible candidates in most of the cases. If your interview is set at the last slot, and the required number of candidates
have already been interviewed or selected internally by the company officials, they would still conduct your interview, expecting something exceptionally good from you, which the others couldn’t deliver, 
and hence there is less chance for you to get selected.

\subsection*{Point2} 
The questions in the exams and interviews are repetitive most of the times. Hence always ask the person who had his interview just before you, and get to know the way you would answer them if they were asked to you. 
There is a very high chance that the same questions would be asked to you. Moreover try as many questions as possible from the placement materials provided by the senior batches. Some of the questions from CISCO, 
Broadcom etc are repetitive from the past years

\subsection*{Point3} 
Once the test for any company is over, discuss all the unsolved question with your friends as soon as you are out of the exam hall, with your friends. Since the interview will probably happen lately on that day or on the next day,
you have time to figure out how to solve those unsolved questions. This is very important, because these same questions could be(so many times it has happened) asked for interview. If you have solved it, you have better chances to impress the interviewer.

\subsection*{Point4} 
Once you have planned to attend a company's placement, do study about what the company is about, what the work profile would be. This is because, firstly that would be a way to impress them once a question is asked
as to what you know about their company. Secondly and most importantly when the interviewer asks you questions, he expects your answer in a certain way related to his thinking and  his line of work. The basic example
is like if you are working in the fab lab in IITB, and a question from a design company is like “Why VLSI scaling is very important” it would not be better to explain him about the technical jargons of device physics,
rather something very incisive, which would give a straight line of thinking as to why the products in his company are also involved in scaling and its necessity.

\subsection*{Point5} Be very patient in answering the questions, and never should it seem to the interviewer that you have prepared for the question beforehand. Try to make it come natural after a round of thinking, 
rather than just spontaneously answering, and going wrong somewhere. The point is the question would be  simple, and when you answer it, very evident points would be missed if you hurry to answer. Those very evident points, 
like in a state machine, when you draw, there would be connections to all the states from the previous ones, but a state from the present state again to the present state would be often missed. But certain companies look for completeness in your thinking. So take your time, solve it, check it, and make sure its complete.

\subsection*{Point6} Prepare in advance regular questions like “Introduce Yourself”, “What are your strengths and Weaknesses”, “What do you expect from a company”, “How was your experience from the past company where you worked” etc. 
These questions may seem very simple, but can at times be decisive on your overall impact. A good preparation on how to attack these kind of questions would give you more confidence for the rest of the interview
 
\fontsize{8pt}{8pt}\selectfont
\subsection*{}
 Note:   This is a placement preparatory guideline made for the Micro-Electronics Branch MTech Students @ IITB, yet, some points and topics can be of help to EE(dept) PG students in general.
 This includes personal experiences of myself and my seniors. The placement experience may vary from person to person, and from time to time. Hence this report may or may not be useful for you 
 depending on your case/cases. Use it at your discretion. I ll not be liable for any wrong message conveyed. 

\end{document}
