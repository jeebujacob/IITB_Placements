\documentclass[a4paper,10pt]{article}
\usepackage{anysize}
\usepackage{amsmath}
\usepackage{amssymb}
\usepackage{graphicx}
\usepackage[left=0.75in, right=0.75in, top=0.75in, bottom=0.75in, includefoot, headheight=13.6pt]{geometry}
\usepackage{color,graphicx}
\usepackage{verbatim}
\usepackage{hyperref}
\usepackage{stmaryrd}
\usepackage{multirow}
\usepackage{textcomp}

\usepackage{verbatim} 


\hypersetup{
    bookmarks=true,         % show bookmarks bar?
    unicode=false,          % non-Latin characters in Acrobat's bookmarks
    pdftoolbar=true,        % show Acrobat's toolbar?
    pdfmenubar=true,        % show Acrobat's menu?
    pdffitwindow=true,      % page fit to window when opened
    pdftitle={CV - Jeebu  Jacob Thomas (Digital)},    % title
    pdfauthor={Jeebu Jacob Thomas},     % author
    pdfsubject={Resume},   % subject of the document
    colorlinks=true,       % false: boxed links; true: colored links
    linkcolor=magenta,        % color of internal links
    citecolor=blue,        % color of links to bibliography
    filecolor=magenta,      % color of file links
    urlcolor=cyan           % color of external links
}

\definecolor{titleColor}{rgb}{0.85, 0.85, 0.85}

\begin{document}
{\qquad \\ \\ \\ \\ \\ \\ \\ \\ \\ \\ \\ \\ \\}
%{1 \\2 \\3 \\4 \\5 \\6 \\7 \\8\\ 9\\ }

% \newpage
%%%%%%%%%%___AREA OF INTEREST____%%%%%%%%
 \colorbox{titleColor}{\parbox{6.5in}{\textbf{AREAS OF INTEREST}}}

 \begin{itemize}
  \setlength{\itemsep}{1pt}

  \item {{  Digital VLSI Design, Processor Design, Hardware design using HDL}}
 \end{itemize}

%%%%%%%%%%___TECHNICAL SKILLS____%%%%%%%%
 \colorbox{titleColor}{\parbox{6.5in}{\textbf{TECHNICAL SKILLS}}}

 \begin{itemize}
  \setlength{\itemsep}{1pt}

  \item \textbf{{Tools / Softwares:}} Cadence-Virtuoso, Xilinx-ISE, ModelSim, NGSPICE, Magic, MATLAB, Keil, Altera Quartus, Labview, Xilinx Chipscope, Xilinx EDK, Microsoft Visual Studio, Git.

  \item \textbf{{Programming Languages:}} VHDL, Verilog, Assembly Language (8085/8051), C/C++, Python, Bluespec System Verilog .
 \end{itemize}

%%%%%%%%%%% ____PROJECTS___ %%%%%%%%%%%
 \colorbox{titleColor}{\parbox{6.5in}{\textbf{PROJECTS / SEMINAR}}}

 \begin{itemize}

\setlength{\itemsep}{1pt}
   \item \textbf{{M.Tech. Project}}\\
        {\textbf{Guide} - Prof. Sachin. B. Patkar}

 \item \textbf{Hardware Accelerator for GF2 Matrix Multiplication }
	Matrix Multiplication being a primitive in applications like cryptography are of prime importance in High Performance Computing. 
	The dark silicon era of VLSI brings in a lot of bottlenecks for software only approaches of matrix multiplication. Hence power efficient
	FPGA based hardware accelerators become an important choice for the cause. The project is to study about different FPGA based cosystems to implement
	an efficient matrix multiplication in GF(2). 
	

 \textbf{\textit{Completed Work:}} 
   32 bit and 64 bit GF2 Matrix Multiplication has been implemented using SIRC(Software Interface for Reconfigurable Computing), a Microsoft based Interface framework
   for hardware-software cosystem, and the results have been compared against the existing software library M4RI. 



%%%%%%%%%%%%%%%%%%%%%%%%%%%%%%%%%%


  \item \textbf{{B.E. Project: Infrared Tracking System using 8051}}  \qquad\qquad\qquad\qquad\qquad\quad \textbf{Software Used}: Keil\\
        {\textbf{Guide} - Asst Prof. Jyothish Chandran, Saintgits College of Engineering, Kottayam.\qquad\qquad\qquad\quad\qquad\qquad\qquad\quad\qquad\qquad\qquad\quad(July-09--May-10)} 
                                                         
	The Infrared tracking system developed is used for tracking slowly moving objects. An IR transmitter mounted on top of the 
	object to be tracked would transmit the signals in all possible directions. Two receivers , each mounted on top of stepper motors
	would rotate, one in the clockwise and the other in the anti-clockwise direction. Once the object is tracked the angle swept by
	the receivers would help trilaterate the exact location of the object, and would display the approximate location  on a LED matrix

  \item \textbf{{R\&D Project: Hardware-Software Co-system in FPGA}}\\
	{\textbf{Guide} - Prof. Sachin. B. Patkar \qquad\qquad\qquad\quad\qquad\qquad\qquad\quad\qquad\qquad\qquad\qquad\qquad\qquad\quad}
	Bluespec Implementation of Partial Basis Builder, Transpose block and Outer Accumulator for Solution of System of linear Equation,
	and Implementation of a 32*32 bit matrix multiplication in EDK using the implemented blocks
	
  \item \textbf{{M.Tech seminar: Study of Xilinx Embedded Development Kit}}\\
	{\textbf{Guide} - Prof. Sachin. B. Patkar \qquad\qquad\qquad\quad\qquad\qquad\qquad\quad\qquad\qquad\qquad\qquad\qquad\qquad\quad}
	Xilinx Embedded Development Kit contains a separate hardware(XPS) and software(SDK) environments each. As an  of a Sparse Matrix-Vector Multiplier

 \end{itemize}

%%%%%%%%%%___RELEVANT COURSES___%%%%%%%%%%%
 \colorbox{titleColor}{\parbox{6.5in}{\textbf{RELEVANT COURSES}}}\\
 
 \begin{tabular}{p{2in}p{2in}p{2.5in}}
%$\circ$ Digital VLSI Design &$\circ$ CMOS Analog VLSI Design &$\circ$ Mixed Signal VLSI Design\\

$\circ$ CMOS Analog VLSI Design &$\circ$ Mixed Signal VLSI Design &$\circ$ Digital VLSI Design\\
$\circ$ VLSI Design LAB &$\circ$ System Design &$\circ$ Microelectronics Simulation Lab\\
$\circ$Foundations of VLSI CAD  &$\circ$ VLSI Technology &$\circ$ Solid State Devices
\end{tabular}\\

%%%%%%%%%%___RELEVANT COURSES___%%%%%%%%%%%
 \colorbox{titleColor}{\parbox{6.5in}{\textbf{RELEVANT COURSES}}}\\
 
 \begin{tabular}{p{2in}p{2in}p{2.5in}}
%$\circ$ Digital VLSI Design &$\circ$ CMOS Analog VLSI Design &$\circ$ Mixed Signal VLSI Design\\

$\circ$ CMOS Analog VLSI Design &$\circ$ Mixed Signal VLSI Design &$\circ$ Digital VLSI Design\\
$\circ$ VLSI Design LAB &$\circ$ System Design &$\circ$ Microelectronics Simulation Lab\\
$\circ$Foundations of VLSI CAD  &$\circ$ VLSI Technology &$\circ$ Solid State Devices
\end{tabular}\\

 %%%%%%%%%%___COURSE PROJ / SEMINAR___%%%%%%%%%%%
 \colorbox{titleColor}{\parbox{6.5in}{\textbf{RELEVANT COURSE PROJECTS}}}

 \begin{itemize}
  \setlength{\itemsep}{1pt}

  \item \textbf{{Implementation of MIPS Architecture}} \textit{(Course: System Design)}\\
	Single cycle implementation of MIPS architecture using 32 bit datapath and Multicycle implementation using 8 bit datapath
	were designed.	Seventeen 32 bit wide instructions were implemented including R-type,I-type and J-type instructions.
   

  \item \textbf{{Design and write synthesizable descriptions of Data processor for 8085 in VHDL}}\\ \textit{(Course: VLSI Design Lab)}\\
	The data processor  implements ALU and performs  all data movement and ALU operations.
	This includes PUSH and POP operations involving the stack processor. The design was coded in
	VHDL and synthesized successfully.
	
  \item \textbf{{Design and write synthesizable descriptions of Instruction processor for 8085 in Verilog}}\\ \textit{(Course: VLSI Design Lab)}\\
	his is an FSM which fetches the next instruction from an address given by the contents of the 16 bit register ‘PC’, 
	and stores it in the instruction register. This circuit will also execute all instructions which update ‘PC’.
	The design was coded in verilog and synthesized successfully.

  \item \textbf{{Design \& Layout of an 3-8 decoder}}   \textit{(Course: Digital VLSI Design)}\\
	A 3-8 decoder with defined total electrical effort was designed in NGSPICE using CMOS logic. Design included forks with
	best possible combinations of inverters to achieve equal delay. Decoder was laid out in Magic and post layout
	simulation results were compared with calculations.

  \item \textbf{{Graph Bipartitioning}} \textit{(Course: Foundations of VLSI CAD)}\\
	Implemented three algorithms for Graph Bipartitioning namely Fiduccia-Mattheyses, Kernighan-Lin Algorithm, and Spectral bisection
	in python.
	
  \item \textbf{{8-point FFT Implementation in Bluespec}} \textit{(Course: Foundations of VLSI CAD)}\\
	Multicycle Implementation of 3 stage pipelined 8-point FFT module was done in Bluespec System Verilog(BSV) and the results
	were verified with the a test function.
	
   \item \textbf{{Design a Gm-C low-pass biquad filter}} \textit{(Course: Mixed Signal VLSI)}\\
 	Gm-C low pass biquad filter which is second order chebyshev filter with maximum passband ripple of 0.5 dB and corner frequency
 	of 100Mhz was designed in UMC 180nm technology. Noise figure less than 20 dB was achieved at the corner frequency. 
   
   \item \textbf{{Design and simulation of switched capacitor filter}} \textit{(Course: Mixed Signal VLSI)}\\
 	A switched capacitor filter with maximum passband attenuation of 1dB at passband corner frequency (20 Khz) and minimum attenuation
 	of 20dB at stopband corner frequency (40KHz) was designed in UMC 180nm technology. Sampling frequency of 220 Khz for switched capacitor was used. 
	
	
   \item \textbf{{Design \& Simulation of 8-bit current steering DAC}}   \textit{(Course: Mixed Signal VLSI Design)}\\
    An 8-bit R-2R DAC was designed in UMC 180nm CMOS technology achieving the worst case absolute value of DNL/INL less than
    $\frac{1}{2}$LSB for given mismatch and variation. Simulation of DNL/INL was carried out for different input codes using Cadence-Virtuoso.
	
  
 \end{itemize}

%%%%%%%%%%___AWARDS AND ACHIEVEMENTS____%%%%%%%%
 \colorbox{titleColor}{\parbox{6.5in}{\textbf{AWARDS AND ACHIEVEMENTS}}}

 \begin{itemize}
 
  \item \textbf{{Technical:}}

  $\circ$ Secured  \textbf{All India Rank 176} with percentile of 99.83 among 1,04,291 candidates in GATE-2010 in Electronics and Communication\\
  $\circ$ Distinguished performance in competitive exams like \textbf{NSEP-2005}(Top 10\%), \textbf{MTS state level scholarship} and District level \textbf{Government scholarships}.
 
  \item \textbf{{Extra curricular:}}\\
  $\circ$ 1st position in PG Kho-Kho, participated from Electrical Department in 2011-12\\
  $\circ$ 1st position in Firodiya Karandak 2009, an intercollegiate multi-talent drama competition at Pune\\
  $\circ$ 3rd position in Purushottam Karandak 2009, an intercollegiate drama competition at Pune\\
  $\circ$ 2nd Position in silent movie making competition in Mood Indigo 2008 at IIT Bombay 
  
  


 \end{itemize}

 %%%%%%%%%%___HOBBY___%%%%%%%%%%%
 \colorbox{titleColor}{\parbox{6.5in}{\textbf{HOBBY}}}

 \begin{itemize}
  \setlength{\itemsep}{1pt}
  \item Kirigami (A paper art form), Badminton, Reading novels
 \end{itemize}

\end{document}
